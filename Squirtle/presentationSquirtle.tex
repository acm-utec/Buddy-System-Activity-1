\documentclass{beamer}
\usepackage[utf8]{inputenc}
\usepackage{graphicx}
\usetheme{Madrid}

% Presentation:
% 1. Pregunta: Realmente Importa dormir? Diego
% 2. Terminos Clave (Ciclo circadiano, cognitividad, performance, etc.) Eduardo done
% 3. Como se puede estudiar el sueño Diego
% 4. Resultados de Estudios Eduardo
% 5. Discusion y Conjeturas Eduardo
% 6. Conclusiones nuestras Diego
% 7. Reflexion Diego y Eduardo

\title{Buddy System Activity 1: Team Squirtle}
\author{Diego Linares \and Eduardo Castro }
\institute{University of Engineering and Technology - UTEC\\ACM-UTEC - IEEE-CS}
\date{September, 2020}

\AtBeginSection[]
{
  \begin{frame}
    \frametitle{Table of Contents}
    \tableofcontents[currentsection]
  \end{frame}
}

\begin{document}

\frame{\titlepage}

\begin{frame}
    \frametitle{Table of Contents}
    \tableofcontents
\end{frame}

\section{Motivational Question}

\begin{frame}{Motivational Question}

Let us ask the following: \\

\medskip

\begin{quotation}
    \textbf{Does sleep really matter?} \\
\end{quotation}

\medskip

As students, it is a common ocurrence that we choose to put work on a higher priority than sleep. We usually perceive that we are more productive that way. \\

\medskip

We have become part of a culture that not only approves of this, it promotes it in order to achieve more, \textbf{allegedly}.
    
\end{frame}

\begin{frame}{Objetive of the Class}

Check how sleep affects people in their cognitive performancec and how can Data Analysis and Computer Science contribute to that. For example, collecting data through \textbf{wearable technology.} \\

\medskip

We want to cover 3 main points:

\begin{itemize}
    \item How does Sleep Work?
    \item Relationship between sleep and IT usage.
    \item How can we sleep better?
\end{itemize}
    
\end{frame}

\section{Keywords}

\begin{frame}{Keywords}
    It's important to have a clear understanding of all keywords in this presentation.

\begin{itemize}
    \item \textbf{Sleep inertia:} Period of time after waking up in which the subject is still suffering from dizziness or sleepiness. Tends to wear out through the first hour of being awake.
    \item \textbf{Circadian rhythms:} Circadian rhythms are 24-hour cycles that are part of the body’s internal clock, running in the background to carry out essential functions and processes.  
    \item \textbf{Sleep homeostasis:} Is a basic principle of sleep regulation. A sleep deficit elicits a compensatory increase in the intensity and duration of sleep, while excessive sleep reduces sleep propensity.
    \item \textbf{Sleep pressure:} Pressure for sleep builds up in our body as our time awake increases. The pressure gets stronger the longer we stay awake and decreases during sleep, reaching a low after a full night of good-quality sleep.
\end{itemize}
\end{frame}

\begin{frame}{Keywords}
\begin{itemize}
     \item \textbf{Sleep debt:} Cummulative hours of sleep that someone is missing after multiple nights of insufficient sleep. Sleep debt is only counted when the subject sleeps less than 6 hours at a time. Mostly compensated when the subject sleeps-in during weekends or off-days. 
    \item \textbf{PANAS scale:}  A survey that can measure your mood.
    \item \textbf{Sleep deprivation:} Term usually referred to the action of not getting sleep, completely supressing it. However it technically stands in for not getting sufficient sleep during a single night. When it happens multiple times it leads to the previously defined sleep debt.
\end{itemize}
\end{frame}


\section{How Can We Study Sleep?}

\subsection{Invasive and Lab Controlled Approaches}

\begin{frame}{Invasive and Lab Controlled Approaches}

Usually more related to test \textbf{sleep deprivation} on subjects. \\

\medskip

Bigger amount of \textbf{control variables}, such as demographic, and excercises to measure cognitive performance. However, the subjects are more conditioned as a result. \\

\medskip

This was used on the paper of \textit{Sleep Debt in Students}. Done in a big West Coast University, on 76 students, by keylogging their phones and asking them to submit surveys and keep a sleeping diary. \\ 

\end{frame}

\begin{frame}{Invasive and Lab Controlled Approaches }

Goal was to measure their multitasking average time as a way to measure focus as well as some other questions:

\begin{itemize}
    \item Impact of sleep on the perception of work pressure and productivity.
    \item Sleep and its relation to multitasking.
    \item Association with Facebook use.
    \item Relation with mood. (Measured with a PANSA scale).
\end{itemize}

\medskip

Sometimes interviews can also be a nice tool to obtain some more insights, as it was done here.
    
\end{frame}

\subsection{Using Wearables and Health Apps}

\begin{frame}{Using Wearables and Health Apps}

These studies helped more identifying sleep patterns and are supported by the ever-increasing amount of people which choose to use these devices for their help. \\

\medskip 

In this case, we checked the \textit{How do we Sleep?} paper which featured the Oura ring. \\ 

\medskip

Already we see a higher amount of people that participate in these studies, and therefore a higher amount of sleep data recorded.

\end{frame}

\begin{frame}{Using Wearables and Health Apps}

A minor setback being that you couldn't get much info. about the cognitive performance of the suer. \\

\bigskip 

But for the scope of its project, it worked pretty well, and it actually had quite a bit of influence on the lifestyle of its users.

\end{frame}

\subsection{Non-Invasive Studies}

\begin{frame}{Non-Invasive Studies}

Have taken popularity since 2004 and especially in the medical field. \\ 

\bigskip

Data collection usually forms part of a mundane or daily activity of the subject. \textbf{Sometimes they don't know the data is being collected.} \\ 

\bigskip 

In this case, we used \textit{Harnessing the web for Population Scale Sensing} done by Microsoft using their search engine Bing. \\

\end{frame}

\begin{frame}{Non-Invasive Studies}

Data was measured through two main features: 

\begin{itemize}
    \item \textbf{Search Box:} Time between the keystrokes of every single word.
    \item \textbf{Results Page:} Time to click on a results after they come out.
\end{itemize}

\bigskip

75 million recordings show how effective these studies are, though a lot of them have to be cut since there is much less control of the subjects. 
    
\end{frame}

\section{Study Results}

\subsection{Invasive and Lab Controlled Approaches}

\subsection{Using Wearables and Health Apps}

\subsection{Non-Invasive Studies}

\begin{frame}{Invasive and Lab Controlled Approaches}

The average sleep duration was 7.9 hours. \\
\bigskip
Many students feel more work pressure the next day when they sleep less. \\
\bigskip
There is a positive correlation between sleep duration and focus duration \\
\bigskip
Sleep debt and time on Facebook. \\
\bigskip
There is a correlation between sleep and mood. \\

\end{frame}

\begin{frame}{Using Wearables and Health Apps}

Most users sleeping less than 7 hours reason for that are: \\
\begin{itemize}
    \item Sleep tracker users sleep less than the normal population. \\
    \item Optimistic self-assessment. \\
    \item It’s hard to distinguish between time in bed and sleep time. \\ 
\end{itemize}

The data show us than lack of sleep is correlated with shorter sleep duration and efficiency. \\

There is also seen to be a high correlation between late bedtime and low sleep score. \\

\end{frame}

\begin{frame}{Non-intrusive studies}
Three functions which model the influence on cognitive performance of time of day.
\begin{itemize}
    \item \textbf{Time of day:} The cognitive performance varies...
    \item \textbf{Time after awakening:}  The first two hours after wake up...
    \item \textbf{Time in bed:} Time in bed: There a relation between performance and time in bed but this relation is less strong...
\end{itemize}
\end{frame}

\section{Discussion and Conjectures}

\begin{frame}{Discussion and Conjectures}
    \begin{itemize}
        \item Ethic issues
        \item Further studies
        \item Measurement
    \end{itemize}
\end{frame}

\section{Conclusions}

\subsection{About the Different Techniques of Collecting Data}

\begin{frame}{About Data Collection Techniques}

There is no \textbf{best way} of taking measures, but there are \textit{do and do nots}, so you can be more accurate to reality.

\begin{itemize}
    \item Make your measures in a way that you are not conditioning the subject.
    \item Some values might be elimintated considering the control variables that you are using.
\end{itemize}

\medskip

These will be taken into account for future statistical studies.
    
\end{frame}

\subsection{About the Sleep Patterns and Facts}

\begin{frame}{About Sleep Patterns and Facts}

Sleep conditions are not the same for everyone, but similar consequences may apply to those who choose to suppress it.

\begin{itemize}
    \item You must take into account your chronotype and how shifted you are.
    \item Massive statistical studies show how lower performance is correlated to insufficient sleep.
\end{itemize}

\medskip
    
IT usage has a 2 way relation with sleep and has some influence on it. Though we can't fully blame it.    
    
\end{frame}

\subsection{Answer to the Motivational Question}

\begin{frame}{Answer to the Motivational Question}

In the grand scheme of things:

\begin{quotation}
    \textbf{Yes, sleep does matter, as it affects cognitive performance and overall physical and psychological health of a person}. \\
\end{quotation}

\bigskip
    
And yes... this applies especially to students.    
    
\end{frame}

\section{Reflection}

\begin{frame}{Diego's Reflection}

It is easy to believe that we are sleep-proof. But it is much harder to deny the evidence that is right there and proves otherwise. \\ 

\bigskip 

This shall be a wake-up call in that sense. \\

\bigskip

Future studies on data analysis will also take into account all the stuff learned while researching for this class. 

\end{frame}

\begin{frame}{Eduardo's Reflection}
    Along this journey...\\
    \bigskip
    The work of so many people trying to understand and prove the importance of sleep...\\
    \bigskip
    Becoming better students...\\
    \bigskip
    Is not just about work day and night... \\
    \bigskip
    Is in our hands.\\
    
\end{frame}

\end{document}
